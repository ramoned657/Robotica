\section{Ecuaciones}
Para realizar ecuaciones, se pueden ayudar mucho de ChatGPT (Como copiar una imagen y que lea la ecuación para dártela en formato \LaTeX) y de que MATLAB, word y algunas páginas te permiten copiar ecuaciones en formato \LaTeX. El modelo en espacio de estados de un robot de dos grados de libertad, el cual se puede ver en el Capítulo 5: Dinámica del Robot en \cite{barrientos2007fundamentos} se expresa como

\begin{equation}
	\label{eq:spaceStateRobot}
	\begin{bmatrix}
		\dot{q} \\
		\ddot{q}
	\end{bmatrix} =
	\begin{bmatrix}
		0 & I \\
		M^{-1}(-C - G)
	\end{bmatrix}
	\begin{bmatrix}
		q \\
		\dot{q}
	\end{bmatrix} +
	\begin{bmatrix}
		0 \\
		M^{-1} B
	\end{bmatrix} u,
\end{equation}
donde:
\begin{itemize}
	\item \( q \) es el vector de posiciones articulares del robot.
	\item \( \dot{q} \) y \( \ddot{q} \) son las velocidades y aceleraciones articulares.
	\item \( M \) es la matriz de inercia.
	\item \( C \) representa las fuerzas centrífugas y de Coriolis.
	\item \( G \) es el vector de fuerzas gravitacionales.
	\item \( B \) es la matriz de entrada de los torques.
	\item \( u \) es el vector de torques aplicados a las articulaciones.
\end{itemize}

Cabe destacar que en \eqref{eq:spaceStateRobot}, la ecuación se referencia después de haberla nombrado y forma parte de la oración, por lo que debe llevar puntos o comas. También al referenciar, debe de estar entre paréntesis con \texttt{eqref}.